\section{Introduction}

The main purpose of the LBNE DAQ system is to read the raw data from the 
Front End Boards (FEB), which are mounted on the Anode Plane Arrays (APA) inside
the cryostat, to build events from the different parts
of the detector and to pass these events on to long term storage.
The Level 3 requirements for this system include\cite{DAQ_REQ}:

\begin{itemize}
\item{LArFD-L3-DAQ-3: The DAQ shall be capable of receiving raw data from a freely running readout from all detector systems.}
\item{LArFD-L3-DAQ-7: The DAQ shall be designed to collect data continuously}
\item{LArFD-L3-DAQ-8: The DAQ shall perform prompt processing of data}
\end{itemize}
The DAQ-7 requirement is relevant mainly to non-beam physics. 
As such, it was left out of the requirements at the time of CD-1, which 
assumed a surface-located Far Detector (FD) with 3 MWE overburden. 
Nonetheless, continuous readout remains a valuable goal that would be desirable to have 
in the final LBNE DAQ system.

Table \ref{tab:rates}, which is taken from Jon Urheim's CD-1 presentation
\cite{DAQ_CD1}
lists
the rates expected in a single Anode Plane Array (APA) for a surface-located
FD.
Assuming that a zero-suppression
algorithm is running in the cold electronics,
the rate is dominated by cosmic ray muons.
Note, however, that common-mode noise,
which may be appreciable,
is not accounted for in this table.

\begin{table}[h]
\begin{tabular}{|p{1.5in}|c|c|c|}
\hline
Process&Rate (kHz/APA)& Samples (per APA) & Avg. Data Rate (Mbps)\\
\hline
Generic 1.4 ms interval (not zero-suppressed)&
0.70&$7.3 \times 10^6$& 61,000\\
\hline
Cosmic Ray Muons & $\approx$ 1 & $2.5 \times 10^4$ & ~300 \\
\hline
Radioactivity: &&& \\
U/Th ($\gamma$'s) & $\approx$ 1&40& 0.48\\
$^{39}$Ar/$^{85}$Kr ($\beta$'s)&40&24&12\\
\hline
Electronics Noise (not common mode) & $\approx 1$ & 15 & 0.2\\
\hline
\end{tabular}
\caption{\label{tab:rates} Expected rates for one APA in a surface-located
FD}
\end{table}

The first (and only) opportunity to test many features of the LBNE APA 
design will be the 35 Ton prototype (35t) to be constructed at FNAL.
The APA(s) of this device are likely to be somewhat smaller than 
those of the full LBNE FD and there will be very little overburden.
The rates listed above are thus only a rough estimate of those to be
expected in the 35t.
Nonetheless, we adopt them for the following design studies.

The key electronics module that needs to be provided for the back-end DAQ is 
one that is capable of reading the data streams from each of the
APA's and concentrating the data down to a smaller number of high-bandwidth 
data streams that are then passed to an event-building network.
This is commonly needed function in modern HEP experiments that has frequently 
been addressed with custom modules built explicitly for a single experiment.
This may require significant development time.
However, the modules produced quickly
become obsolete, as available networking technology progresses.
This limits the desirability of reusing such modules in subsequent experiments.

The SLAC Research Electronics Group (REG) has developed a solution to this 
obsolescence problem 
by producing a set of modules, together with firmware and software, that can
be adapted for use in multiple experiments.
The development costs are then leveraged over multiple experiments, allowing 
each of them to benefit from the latest networking hardware, at a significant 
reduction in development costs. 
This ``DAQ toolkit'' uses the modern Advanced Telecommunications Architecture 
(ATCA)for its physical structure. 
The key element of the system is the Reconfigurable Cluster Element (RCE).
The RCE is based on a Virtex  ``System on a Chip'', which could be 
either a Virtex 5 chip (RCE Gen 2) or a Xynq chip (RCE Gen 3).
A single board combining several of these RCE's can handle very high bandwidths measured
in the 100's of Gigabits/second. 
This system has been adopted in several HEP experiments already, and will likely 
be adopted by more in the future. 
The REG is continuing to develop and support new generations 
of the toolkit to take advantage of new networking equipment as it becomes available.

We are proposing to make use of this toolkit in the DAQ systems to be produced
for the LBNE 35 ton prototype and the full Far Detector.
The bandwidth available in the current generation of the toolkit far
exceeds that of the baseline system based on the Nova Data Concentrator 
Module (DCM).
The increased flexibility afforded by this extra bandwidth may be highly valuable to 
ensuring LBNE success.
Furthermore, leveraging the work already done by the REG as well as benefitting from
their support in the future, will provide many benefits to LBNE and may reduce
the development costs.