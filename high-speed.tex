The baseline DAQ design for the full LBNE FD envisions a high degree of zero-suppression 
occuring in the in-cryostat electronics and a rather modest total data rate (about
300 Mb/second/APA) being transmitted to the backend DAQ.
This rate would be spread over roughly 20 cables transmitting at a maximum rate of 
20 Mbit/second.
This approach has a number of advantages in that the low transmission speed is likely 
achievable with standard cables and issues of impedance matching are less of a concern.
However, it does rely on the zero-suppression algorithm being nearly perfect and has 
little headroom in case unexpected backgrounds or noise are present.

Raising the transmission bandwidth between FEB and DAQ thus seems desirable. 
An extreme case of this would be to go to the highest possible bandwidth carriable over
copper wires.
With current technology, this rate is about 5 Gbit/second. 
Cables capable of carrying this rate over distances of about 5 meters are commercially
available.