The baseline design for the LBNE DAQ \cite{DAQ_CD1}
is based on Data Concentrator Modules (DCMs) 
that were originally designed for use in the Nova experiment.
Each of these modules takes 64 input data data streams of 20 Mbps and 
combine them into a single stream of 1 Gbps. 
The total throughput of a DCM module is limited to 60 Mbytes/second.
A single DCM module is thus able to handle enough data to readout the
entire 35 ton prototype at the expected rate of 300 Mbit/second/APA,
although without a large amount of headroom.
The main advantage of an RCE-based system is a vast improvement
in the available bandwidth, which adds signficant flexibility in 
the handling of the raw datastream.
For example, much high noise rates could be handled.
As an extreme example, one could consider reading non-zero-suppressed data 
into the DAQ system.